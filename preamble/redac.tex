\newcommand{\commentOC}[1]{{\small\color{blue}$\big[$OC: #1$\big]$}}
\newcommand{\commentOCf}[1]{{\small\color{blue}{\selectlanguage{french}$\big[$OC : #1$\big]$}}}
\newcommand{\commentYM}[1]{{\small\color{red}$\big[$YM: #1$\big]$}}
\newcommand{\commentYMf}[1]{{\small\color{red}{\selectlanguage{french}$\big[$YM : #1$\big]$}}}

\nottoggle{LCpres}{
%style is plain by default (italic text), %http://tex.stackexchange.com/questions/144653/italicizing-of-amsthm-package. TODO: check ntheorem, which https://ctan.org/pkg/amsmath recommends.
	\newtheorem{theorem}{Theorem}
	\newtheorem{lemma}{Lemma}
	\theoremstyle{definition}
		\newtheorem{definition}{Definition}
	\theoremstyle{remark}
		\newtheorem{examplex}{Example}
		\newtheorem{remarkx}{Remark}
%Trickery allowing use of \qedhere and the like.
	\newenvironment{example}{
		\pushQED{\qed}\renewcommand{\qedsymbol}{$\triangle$}\examplex
	}{
		\popQED\endexamplex
	}
	\newenvironment{remark}{
		\pushQED{\qed}\renewcommand{\qedsymbol}{$\triangle$}\remarkx
	}{
		\popQED\endremarkx
	}
}{
}
%TODO \crefname{axiom}{axiom}{axioms}%might be needed for workaround bug in cref when defining new theorems?
\crefname{examplex}{example}{examples}% I wonder why this is unnecessary in case of singular

%I find these settings useful in draft mode.
%Which line breaks are chosen: accept worse lines, therefore reducing risk of overfull lines. Default = 200.
	\tolerance=2000
%Accept overfull hbox up to...
	\hfuzz=2cm
%Reduces verbosity about the bad line breaks.
	\hbadness 5000
%Reduces verbosity about the underful vboxes.
	\vbadness=1300

\bibliographystyle{abbrvnat}

%TODO test the brackets.
%\newcommand{\llbracket}{\lbrack\!\lbrack}
%\newcommand{\rrbracket}{\rbrack\!\rbrack}

%2212 Minus Sign
\newunicodechar{−}{\ifmmode{-}\else\textminus\fi}
%These symbols can only be used in math mode, for I found no text equivalent.
%03B3 Greek Small Letter Gamma
\newunicodechar{γ}{\gamma}
%03B4 Greek Small Letter Delta
\newunicodechar{δ}{\delta}
%2115 Double-Struck Capital N
\newunicodechar{ℕ}{\mathbb{N}}
%211D Double-Struck Capital R
\newunicodechar{ℝ}{\mathbb{R}}
%21CF Rightwards Double Arrow with Stroke
\newunicodechar{⇏}{\nRightarrow}
%21D2 Rightwards Double Arrow
\newunicodechar{⇒}{\Rightarrow}
%21D4 Left Right Double Arrow
\newunicodechar{⇔}{\Leftrightarrow}
%2227 Logical And
\newunicodechar{∧}{\land}
%2228 Logical Or
\newunicodechar{∨}{\lor}
%2229 Intersection
\newunicodechar{∩}{\cap}
%222A Union
\newunicodechar{∪}{\cup}
%2260 Not Equal To
\newunicodechar{≠}{\neq}
%2264 Less-Than or Equal To
\newunicodechar{≤}{\leq}
%2265 Greater-Than or Equal To
\newunicodechar{≥}{\geq}
%227B Succeeds
\newunicodechar{≻}{\succ}
%2281 Does Not Succeed
\newunicodechar{⊁}{\nsucc}
%22EB Does Not Contain As Normal Subgroup
\newunicodechar{⋫}{\ntriangleright}
%25A1 White Square
\newunicodechar{□}{\Box}
%25B7 White Right-Pointing Triangle
%TODO test \rhd; \ntrianglerighteq from amssymb?
\newunicodechar{▷}{\triangleright}
%27E6 Mathematical Left White Square Bracket – there’s also \llbracket from stmaryrd
\newunicodechar{⟦}{\text{\textlbrackdbl}}
%27E7 Mathematical Right White Square Bracket – there’s also \rrbracket from stmaryrd
\newunicodechar{⟧}{\text{\textrbrackdbl}}
%27FC Long Rightwards Arrow from Bar
\newunicodechar{⟼}{\longmapsto}
%2AB0 Succeeds Above Single-Line Equals Sign
\newunicodechar{⪰}{\succeq}
%301A Left White Square Bracket
\newunicodechar{〚}{\textlbrackdbl}
%301B Right White Square Bracket
\newunicodechar{〛}{\textrbrackdbl}
%→ is defined by default as \textrightarrow, which is invalid in math mode. Same thing for the three other commands. I redefine those four using \DeclareUnicodeCharacter instead of \newunicodechar because the latter warns about the previous definition.
%→ Rightwards Arrow
\DeclareUnicodeCharacter{2192}{\ifmmode\rightarrow\else\textrightarrow\fi}
%¬ Not Sign
\DeclareUnicodeCharacter{00AC}{\ifmmode\lnot\else\textlnot\fi}
%… Horizontal Ellipsis
\DeclareUnicodeCharacter{2026}{\ifmmode\dots\else\textellipsis\fi}
%× Multiplication Sign
\DeclareUnicodeCharacter{00D7}{\ifmmode\times\else\texttimes\fi}

% For the majority relation
\newcommand{\Maj}{\text{Maj}}

