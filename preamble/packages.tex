%Setting pdfgentounicode to one permits (together with package glyphtounicode) to copy eg x ⪰ y iff v(x) ≥ v(y) from pdf to unicode data. 
	\pdfgentounicode=1 
	\input{glyphtounicode}
%Latin Modern has more glyphs than Computer Modern, such as diacritical characters, and permits copy from resulting PDF. fntguide commands to load the font before fontenc, to prevent default loading of cmr.
	\usepackage{lmodern}
%Encode resulting accented characters correctly in resulting PDF, permits copy from PDF.
	\usepackage[T1]{fontenc}
%UTF8 seems to be the default in recent TeX installations, but not all. https://tex.stackexchange.com/a/370280
	\usepackage[utf8]{inputenc}
%Provides \newunicodechar for easy definition of supplementary UTF8 characters such as → or ≤ for use in source code.
	\usepackage{newunicodechar}
%Text Companion fonts, much used together with CM-like fonts. Provides \texteuro and other similar commands for text mode characters such as \textminus, \textrightarrow, \textlbrackdbl.
	\usepackage{textcomp}
%Solves bug in lmodern, https://tex.stackexchange.com/a/261188. (TODO see if useful?)
	%Declare `cmex` to be arbitrary scalable.
	\DeclareFontShape{OMX}{cmex}{m}{n}{
	  <-7.5> cmex7
	  <7.5-8.5> cmex8
	  <8.5-9.5> cmex9
	  <9.5-> cmex10
	}{}
	\SetSymbolFont{largesymbols}{normal}{OMX}{cmex}{m}{n}
	\SetSymbolFont{largesymbols}{bold}  {OMX}{cmex}{m}{n}
%More symbols available in bold version, see https://github.com/latex3/latex2e/issues/71.
	\DeclareFontShape{OMX}{cmex}{bx}{n}{%
	   <->sfixed*cmexb10%
	   }{}
	\SetSymbolFont{largesymbols}{bold}{OMX}{cmex}{bx}{n}
%Warn about missing characters.
	\tracinglostchars=2
%\usepackage{booktabs}
%TODO should be loaded!
%\usepackage{calc}
%\usepackage{tabularx}
%Provides \addtocmd, \patchcmd, \newtoggle commands.
	\usepackage{etoolbox} 
%mathtools requires amsmath, which is anyway considered a basic, mandatory package nowadays (Grätzer, More Math Into LaTeX).
	\usepackage{amsmath}
%Seems better to load mathtools before babel.
%mathtools fixes some bugs in amsmath; permits to hide (non manual) tags that are never referenced; provides \refeq for truthful typesetting of manual tag references and \DeclarePairedDelimiter.
	\usepackage{mathtools} 
	\mathtoolsset{showonlyrefs, showmanualtags}
%Package frenchb asks to load natbib before babel-french. Package hyperref asks to load natbib before hyperref.
	\usepackage{natbib}
%Language options ([french, english]) should be on the document level (last is main).
%TODO do not remove babel, which beamer uses (beamer uses the \translate command for the appendix); but french can be removed.
%	\usepackage{babel}
%	\frenchbsetup{AutoSpacePunctuation=false}
%\usepackage{listings}
	%\lstset{tabsize=2}
%I favor acro over acronym because the former is more recently updated (2017 VS 2015 at time of writing); has a longer user manual (about 40 pages VS 6 pages if not counting the example and implementation parts); has a command for capitalization; and acronym suffers a nasty bug when ac used in section, see http://tex.stackexchange.com/questions/103483/strange-packages-interaction-acronyms-silence-hyperref (though this might be the fault of the silence package and might be solved in more recent versions, I do not know).
	%\usepackage{acro}

\newtoggle{LCpres}
\togglefalse{LCpres}

\iftoggle{LCpres}{
	%I favor fmtcount over nth because it is loaded by datetime anyway; and fmtcount warns about possible conflicts when loaded after nth.
	\usepackage{fmtcount}
	%For nice input of date of presentation. Must be loaded after the babel package. Has possible problems with srcletter: https://golatex.de/verwendung-von-babel-und-datetime-in-scrlttr2-schlaegt-fehlt-t14779.html.
	\usepackage[nodayofweek]{datetime}
}{
}
%I do not use option pdfusetitle (which must be introduced here and not in hypersetup) in articles because the title often has some asterisk or other supplement that must not appear in the pdf metadata. For presentations, Beamer implicitely uses the pdfusetitle option.
	\usepackage{hyperref}
	%I hide links for those which can be recognized as links by the reader without highlighting, to not distract the reader. urlbordercolor is used both for \url (and \doi) and for \href, thus might need to be removed. TODO test in presentation.
	\hypersetup{linkbordercolor={1 1 1}, citebordercolor={1 1 1}, urlbordercolor={1 1 1}}
	%hyperref doc says: “Package bookmark replaces hyperref’s bookmark organization by a new algorithm (...) Therefore I recommend using this package”.
	\usepackage{bookmark}
%Need to invoke hyperref explicitly to link to line numbers: \hyperlink{lintarget:mylinelabel}{\ref*{lin:mylinelabel}}, with \ref* to disable automatic link. Also see https://tex.stackexchange.com/questions/428656/external-cross-reference-to-a-line-number-using-lineno-and-xr-hyper for referencing lines from another document.
	%\usepackage{lineno}
	%\newcommand{\llabel}[1]{\hypertarget{lintarget:#1}{}\linelabel{lin:#1}}
	%\setlength\linenumbersep{9mm}
%For complex authors blocks. Seems like authblk wants to be later than hyperref, but sooner than silence.
	\nottoggle{LCpres}{
		\usepackage{authblk}
		\renewcommand\Affilfont{\small}
	}{
	}
%I do not use floatrow, because it requires an ugly hack for proper functioning with KOMA script (see scrhack doc). Instead, the following command centers all floats, and I manually place my table captions above and figure captions below their contents (https://tex.stackexchange.com/a/3253).
	\makeatletter
	\g@addto@macro\@floatboxreset\centering
	\makeatother
%Permits to customize enumeration display and references
	%\nottoggle{LCpres}{
		%\usepackage{enumitem} %follow enumerate by a string saying how to display enumeration
	%}{
	%}
%Provides \Cen­ter­ing, \RaggedLeft, and \RaggedRight and en­vi­ron­ments Cen­ter, FlushLeft, and FlushRight, which al­low hy­phen­ation. 
	%\usepackage{ragged2e}
%To typeset units by closely following the “official” rules.
	%\usepackage[strict]{siunitx} %[expproduct=tighttimes, decimalsymbol=comma] ou (plus récent ?) [round-mode=figures, round-precision=2, scientific-notation = engineering]
%My favorites bibliographystyle’s provide doi’s. The package doi turns those into urls. However, it uses old-style dx.doi url (see 3.8 DOI system Proxy Server technical details, “Users may resolve DOI names that are structured to use the DOI system Proxy Server (http://doi.org (preferred) or http://dx.doi.org).”, https://www.doi.org/doi_handbook/3_Resolution.html). The patch solves this.
	\usepackage{doi}
	\makeatletter
	\patchcmd{\@doi}{dx.doi.org}{doi.org}{}{}
	\makeatother
%Makes sure upper case greek letters are italic as well.
	\usepackage{fixmath}
%Provides \mathbb; obsoletes latexsym (see http://tug.ctan.org/macros/latex/base/latexsym.dtx). Relatedly, \usepackage{eucal} to change the mathcal font and \usepackage[mathscr]{eucal} (apparently equivalent to \usepackage[mathscr]{euscript}) to supplement \mathcal with \mathscr. This last option is not very useful as both fonts are similar, and the intent of the authors of eucal was to provide a replacement to mathcal (see doc euscript). Also provides \mathfrak for supplementary letters.
	\usepackage{amsfonts}
%Provides a beautiful (IMHO) \mathscr and really different than \mathcal, for supplementary uppercase letters. TODO drawback is that no bold version!?
	\usepackage{mathrsfs}
%Multiple means to produce bold math: \mathbf, \boldmath (defined to be \mathversion{bold}, see fntguide), \pmb, \boldsymbol (all legacy, from LaTeX base and AMS), \bm (the most recommended one), \mathbold from package fixmath (I don’t see its advantage over \boldsymbol).
%“The \boldsymbol command is obtained preferably by using the bm package, which provides a newer, more powerful version than the one provided by the amsmath package. Generally speaking, it is ill-advised to apply \boldsymbol to more than one symbol at a time.” — AMS Short math guide. “If no bold font appears to be available for a particular symbol, \bm will use ‘poor man’s bold’” — bm. It is “best to load the package after any packages that define new symbol fonts” – bm. bm defines \boldsymbol as synonym to \bm.  \boldmath accesses the correct font if it exists; it is used by \bm when appropriate. See https://tex.stackexchange.com/a/10643 for some difficulties with \bm.
%TODO check how to make it warn when no bold or poor bold.
	\usepackage{bm}
%amsthm corrects the spacing of proclamations, allows for theoremstyle, and is considered a basic, mandatory package nowadays (Grätzer, More Math Into LaTeX).
	\usepackage{amsthm}
%Provides \cref. Unfortunately, cref fails when the language is French and referring to a label whose name contains a colon (https://tex.stackexchange.com/questions/83798/cleveref-varioref-missing-endcsname-inserted). cleveref should go “laster” than hyperref.
	\usepackage{cleveref}
%\usepackage{tikz}
	%\usetikzlibrary{babel, matrix, fit, plotmarks, calc, trees, shapes.geometric, positioning, plothandlers, arrows, shapes.multipart}
%Vizualization, on top of TikZ
	%\usepackage{pgfplots}
	%\pgfplotsset{compat=1.14}
\usepackage{graphicx}
	\graphicspath{{graphics/}}

%Provides \print­length{length}, useful for debugging.
	%\usepackage{printlen}
	%\uselengthunit{mm}
%Provides \NewDocumentCommand and similar commands intended as replacement of \newcommand in LaTeX3 (for package authors? https://tex.stackexchange.com/questions/98152/always-use-newdocumentcommand-instead-of-newcommand TODO).
	\usepackage{xparse}

\iftoggle{LCpres}{
	%“fixes the frame num­ber­ing in beamer when us­ing an ap­pendix such that the slides from the ap­pendix are not counted in the to­tal frame num­ber of the main part of the doc­u­ment”.
		\usepackage{appendixnumberbeamer}
	%I have yet to see anyone actually use these navigation symbols – this command removes them.
		\setbeamertemplate{navigation symbols}{}
	\usepackage{preamble/beamerthemeParisFrance}
}{
}
%tikzposter-specific
%remove \usepackage{ragged2e}: causes 1=1 to be printed in the middle of the poster. (Anyway prints a warning about those characters being missing.)
%put [french, english] options next to \usepackage{babel} to avoid warning

%TODO Compare $\sum$ and $\boldsymbol{\sum}$ and ${\mathversion{bold} \sum}$. Check whether nag should be loaded first (it says strange things).

